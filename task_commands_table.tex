\newcommand{\ra}{\rightarrow}
\newcommand{\tb}{\textbf}
\definecolor{gray}{rgb}{0.8,0.8,0.8}

% \begin{pycode}
% import tex_table
% table = tex_table.get_table(latex=True)
% \end{pycode}

\renewcommand{\hline}{\noalign{\hrule height 1.2pt}}
\begin{table}[H]
  \small
  \centering
  % \begin{tabular}{>{\setlength\hsize{.5\hsize}} X >{\setlength\hsize{1.5\hsize}} X}
  \rowcolors{15}{gray}{gray}
  \begin{tabular}{|r*{6}{|l}}
  \hline
№  & Адрес & Команда & Мнемоника   & Комментарий \\ \hline
1  & 359   &036E     & X1          & Адрес начала массива \\ \hline
2  & 35A   &0200     & X2          & Адрес текущего элемента \\ \hline
3  & 35B   &4000     & M           & Длина массива \\ \hline
4  & 35C   &0200     & R           & Результат программы \\ \hline
5  & 35D   &0200     & CLA         & \multirow{2}{*}{очистка результата} \\ \cline{1-4}
6  & 35E   &EEFD     & ST R   & \\ \hline
7  & 35F   &AF05     & LD \#5      & \multirow{2}{*}{Загрузка длины массива} \\ \cline{1-4}
8  & 360   &EEFA     & ST M   & \\ \hline
9  & 361   &4EF7     & ADD X1  & \\ \cline{1-4}
10 & 362   &EEF7     & ST X2   & \multirow{-2}{*}{Установка адреса текущего элемента + 1} \\ \hline
11 & 363   &ABF6     & LD (X2)- & Загрузка элемента из массива с декрементом\\ \hline
12 & 364   &0480     & ROR         & Деление на 2 с остатком \\ \hline
13 & 365   &F405     & BHIS IP+5   & continue, если не делится на 2\\ \hline
14 & 366   &0480     & ROR         & Деление ещё на 2 с остатком \\ \hline
15 & 367   &F403     & BHIS IP+3   & continue, если не делится на 2\\ \hline
16 & 368   &0400     & ROL         & \\ \hline
17 & 369   &0400     & ROL         & \\ \hline
18 & 36A   &AAF1     & LD (R)+ & Увеличение результата на 1\\ \hline
19 & 36B   &835B     & LOOP \$35B  & \\ \cline{1-4}
20 & 36C   &CEF6     & JUMP (IP-10)& \multirow{-2}{*}{Остаёмся в цикле, если ещё есть элементы} \\ \hline
\hiderowcolors
21 & 36D   &0100     & HLT         & Останов\\ \hline
22 & 36E   &8369     & Y1          & Элемент массива\\ \hline
23 & 36F   &CE00     & Y2          & Элемент массива\\ \hline
24 & 370   &0003     & Y3          & Элемент массива\\ \hline
25 & 371   &F200     & Y4          & Элемент массива\\ \hline
26 & 372   &F201     & Y5          & Элемент массива \\\hline
  \end{tabular}
\end{table}
