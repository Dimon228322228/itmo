%!Tex Root=../report.tex
\newcommand\maxSign{2^{15}-1}
\newcommand\minSign{-2^{15}}
\newcommand\aConst{-2883}
\newcommand\bConst{122}
\newcommand\minToConstInterv{[\minSign; \aConst)}
\newcommand\constInterval{[\aConst; 0]}
\newcommand\zeroToMax{(0;\maxSign]}

\subsection{Назначение программы}
Программа реализует функцию:
\begin{align*}
    R &= (f(Z) + 1) - (f(Y) - 1) + (f(X) + 1) \\
      &= f(Z) - f(Y) + f(X) + 3
\end{align*}


Подпрограмма реализует функцию:
\begin{equation*}
f(x) =
\begin{cases}
    x \cdot 3 + B & \text{if } x > 0 \text{ or } x < -2883_{10} \\
    -2883_{10} &         \text{if } -2883_{10} \le x \le 0
\end{cases}
\end{equation*}

% \subsection{Расположение в памяти БЭВМ программы, исходных данных и результатов}
% \begin{itemize}
%     \item 206-208 --- переменные
%     \item 209 --- результат
%     \item 700 --- константа A
%     \item 701 --- константа B
% \end{itemize}

% \subsection{Область представления данных}
% A, B, $X$, $Y$, $Z$, R – целые знаковые шестнадцатеричные числа

% \subsection{Область допустимых значений}
% % A = F4BD$_16$ = $-2883_{10}$ \\
% % B = 007A$_16$ = $122_{10}$ \\
% % \\
% Результат: $\minSign \le R \le \maxSign$ \\
% $\minSign \le (f(Z) + 1) - (f(Y) - 1) + (f(X) + 1) \le \maxSign$
% \begin{flalign*}
%     &
%     \begin{cases}
%         \frac{\minSign}{3} \le f(Z) + 1 \le \frac{\maxSign}{3} \\
%         \frac{\minSign}{3} \le f(Y) - 1 \le \frac{\maxSign}{3} \\
%         \frac{\minSign}{3} \le f(X) + 1 \le \frac{\maxSign}{3} \\
%     \end{cases}
%     && \\
%     &
%     \begin{cases}
%         \frac{\minSign}{3} - 1\le f(Z) \le \frac{\maxSign}{3} - 1 \\
%         \frac{\minSign}{3} + 1 \le f(Y) \le \frac{\maxSign}{3} + 1\\
%         \frac{\minSign}{3} - 1\le f(X) \le \frac{\maxSign}{3} - 1 \\
%     \end{cases}
%     && \\
%     &
%     \begin{cases}
%         \frac{\minSign}{3} - 1\le 3\cdot Z + \bConst \le \frac{\maxSign}{3} - 1 \\
%         \frac{\minSign}{3} + 1 \le 3\cdot Y + \bConst \le \frac{\maxSign}{3} + 1\\
%         \frac{\minSign}{3} - 1\le 3\cdot X + \bConst \le \frac{\maxSign}{3} - 1 \\
%     \end{cases}
%     && \\
%     &
%     \begin{cases}
%         \frac{\minSign}{3} - 1 - \bConst \le 3\cdot Z \le \frac{\maxSign}{3} - 1 - \bConst \\
%         \frac{\minSign}{3} + 1 - \bConst \le 3\cdot Y \le \frac{\maxSign}{3} + 1 - \bConst \\
%         \frac{\minSign}{3} - 1 - \bConst \le 3\cdot X \le \frac{\maxSign}{3} - 1 - \bConst \\
%     \end{cases}
%     && \\
%     &
%     \begin{cases}
%         -3681 \le Z \le 3599 \\
%         -3681 \le Y \le 3600 \\
%         -3681 \le X \le 3599 \\
%     \end{cases}
%     && \\
% \end{flalign*}

% Результат, при подстановке чисел, входящих в ОДЗ, помещается в разрядную сетку БВМ. \\\\

% Если рассматривать все случаи, то получаем следующее: \\

% Учитываем возможное переполнение при сложении/вычитании результатов подпрограмм. \\
% \begin{flalign*}
%     & \minSign           \le (f(Z)+1)-(f(Y)-1)   \le \maxSign && \\
%     & \minSign           \le f(Z)-f(Y)+2         \le \maxSign && \\
%     & \minSign           \le f(Z)-f(Y)           \le \maxSign - 2 && \\
%     & \frac{\minSign}{2} \le \frac{f(Z)-f(Y)}{2} \le \frac{\maxSign - 2}{2} && \\
%     &
%     \begin{cases}
%         f(Z) \in [-16384; 16382] \\
%         f(Y) \in [-16382; 16384]
%     \end{cases}
%     &&
% \end{flalign*}
% Учитываем возможное переполнение при прибавлении/вычитании 1 к результату подпрограммы.
% С учётом предыдущих ограничений. \\
% \begin{flalign*}
%     &
%     \begin{cases}
%         \minSign \le f(X) \le \maxSign - 1 \\
%         -16384 \le f(Z) \le 16382 - 1 \\
%         -16382 + 1 \le f(Y) \le 16384
%     \end{cases}
%     && \\
%     &
%     \begin{cases}
%         \frac{\minSign}{3} \le X \le \frac{\maxSign - 1 - \bConst}{3} \\
%         \frac{-16384}{3}   \le Z \le \frac{16382 - 1 - \bConst}{3} \\
%         \frac{-16381}{3}   \le Y \le \frac{16384 - \bConst}{3}
%     \end{cases}
%     && \\
%     &
%     \begin{cases}
%         -10922 \le X \le 10880  \\
%         -5461  \le Z \le 5419  \\
%         -5460  \le Y \le 5420  \\
%     \end{cases}
%     &&
% \end{flalign*}
% \begin{enumerate}
%     \item Если $X, Y, Z \in [\minSign; \aConst) \cup (0;\maxSign]:$
%         \begin{flalign*}
%              & R = 3\cdot(Z - Y + X) + \bConst + 3 && \\
%              & \minSign \le 3\cdot(Z - Y + X) + 125 \le \maxSign && \\
%              & -10964   \le (Z - Y + X)             \le 10880 && \\
%              & -3654    \le \frac{Z - Y + X}{3}     \le 3626 && \\
% 			\
%              & Z, -Y, X \in [-3654; 3626] && \\
%              & Z, -Y, X \in [-3654; \aConst) \cup (0; 3626] && \\
%              & \text{гарантирует отсутсвие переполнения } &&
%         \end{flalign*}
%     \item Если $X \in [\aConst; 0]:$
% 		\begin{flalign*}
%              & R = 3\cdot(Z - Y) \aConst + 3 && \\
%              & R = 3\cdot(Z - Y) -2880 && \\
%              \ 
%              & \minSign \le 3\cdot(Z - Y) -2880 \le \maxSign && \\
%              & -9962 \le (Z - Y)         \le 10922 && \\
%              & -4981 \le \frac{Z - Y}{2} \le 5461 && \\
%              \
%              & Z, -Y \in [-4981; 5461] && \\
%              & \text{таким образом, учитывая ограничения:} &&\\
%              & Z \in [-4981; \aConst) \cup (0; 5419] && \\
%              & Y \in [-5420; 0) \cup (2883; 4981] &&
%         \end{flalign*}
%     \item Если $Y \in [\aConst; 0]:$
% 		\begin{flalign*}
%              & \text{То же, что и в предыдущем} && \\
%              & \ldots && \\
%              & Z, X \in [-4981; 5461]  && \\
%              \
%              & \text{ таким образом, учитывая ограничения:} &&\\
%              & Z \in [-4981; \aConst) \cup (0; 5419] && \\
%              & X \in [-4981; \aConst) \cup (0; 5461] &&
%         \end{flalign*}
%     \item Если $Z \in [\aConst; 0]:$
% 		\begin{flalign*}
%              & \text{То же, что и в предыдущем} && \\
%              & \ldots && \\
%              \
%              & \text{таким образом, учитывая ограничения:} &&\\
%              & X \in [-4981; \aConst) \cup (0; 5461] && \\
%              & Y \in [-5420; 0) \cup (2883; 4981] &&
%         \end{flalign*}
%     \item Если $X, Y \in [\aConst; 0]:$
% 		\begin{flalign*}
%             & R = 3\cdot Z + \bConst + 3 && \\
%             & \ldots && \\
%             & -10964 \le Z \le 10880 && \\
%             \
%             & \text{ таким образом, учитывая ограничения:} &&\\
%             & Z \in [-5461; 5419] \And Z \not\in \constInterval &&
%         \end{flalign*}
%     \item Если $Z, Y \in [\aConst; 0]:$
% 		\begin{flalign*}
%              & X \in [-10922; 10880] \And X \not\in \constInterval &&
%         \end{flalign*}
%     \item Если $X, Z \in [\aConst; 0]:$
% 		\begin{flalign*}
%             & R = -3\cdot Y - \bConst + 2\cdot(\aConst) + 3 && \\
%             & \ldots && \\
%             & -12884 \le Y \le 8961 &&\\
%             & \text{ таким образом, учитывая ограничения:} &&\\
%             & \minSign \le 3\cdot Y + \bConst \le \maxSign &&\\
%             & \ldots && \\
%             & Y \in [-10922; 5420] \And Y \not\in \constInterval &&
%         \end{flalign*}
%     \item Если $X, Y, Z \in [\aConst; 0]:$
% 		\begin{flalign*}
%             & R = \aConst + 3 = -2880 && \\
%             & \text{переполнения нет}
%         \end{flalign*}
% \end{enumerate}

% Итоговое ОДЗ, если рассматривать все случаи: \\
% \begin{flalign*}
%     &
%     \left[
%     \begin{gathered}
%         \begin{cases}
%            X \in [-3654; \aConst) \cup (0; 3626] \\
%            Y \in [-3626; \aConst) \cup (0; 3654] \\
%            Z \in [-3654; \aConst) \cup (0; 3626] \\
%         \end{cases} \hfill \hfill \\
%         \begin{cases}
%            X \in \constInterval \\
%            Y \in [-5420; 0) \cup (2883; 4981] \\
%            Z \in [-4981; \aConst) \cup (0; 5419] \\
%         \end{cases} \hfill \\
%         \begin{cases}
%            X \in [-4981; \aConst) \cup (0; 5461] \\
%            Y \in \constInterval \\
%            Z \in [-4981; \aConst) \cup (0; 5419] \\
%         \end{cases} \hfill \\
%         \begin{cases}
%            X \in [-4981; \aConst) \cup (0; 5461] \\
%            Y \in [-5420; 0) \cup (2883; 4981] \\
%            Z \in \constInterval \\
%         \end{cases} \hfill \\
%         \begin{cases}
%            X \in \constInterval \\
%            Y \in \constInterval \\
%            Z \in [-5461; \aConst) \cup (0; 5419] \\
%         \end{cases} \hfill \\
%         \begin{cases}
%            X \in [-10922; \aConst) \cup (0; 10880] \\
%            Y \in \constInterval \\
%            Z \in \constInterval \\
%         \end{cases} \hfill \\
%         \begin{cases}
%            X \in \constInterval \\
%            Y \in [-10922; \aConst) \cup (0; 5420] \\
%            Z \in \constInterval \\
%         \end{cases} \hfill \\
%         \begin{cases}
%            X \in \constInterval \\
%            X \in \constInterval \\
%            Z \in \constInterval \\
%         \end{cases}
%         \hfill
%     \end{gathered}
%     \right.
%     &
% \end{flalign*}

% \subsection{График функции}
% \pgfmathsetmacro{\A}{-2883}
% \pgfmathsetmacro{\B}{122}

% \begin{tikzpicture}[font=\tiny]
% \begin{axis}[
%         axis y line=center,
%         axis x line=middle, 
%         axis on top=true,
%         xmax=1000,
%         ymax=1000,
%         extra y ticks = {\A},
%         extra y tick style = {y tick label style={right, xshift=0.25em}}
% ] 

% % A should be less than 0.
% \addplot[domain=-3200:\A, red, samples=10,
%     ] {x * 3 + \B};
% \addplot[domain=\A:0, red, samples=10] {\A};
% \addplot[domain=0:40, red, samples=10] {x * 3 + \B};
% \end{axis}
% \end{tikzpicture}
