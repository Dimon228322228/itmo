\subsection{Назначение}
Программа считает количество элементов массива делящихся на 4.

\subsection{Область представления данных}
\begin{itemize}
	\item X1 --- адрес первого элемента массива, 11-разрядное беззнаковое число.
	\item X2 --- адрес последнего элемента массива, 11-разрядное беззнаковое число.
	\item M --- длина массива, 7-разрядное, беззнаковое число. 
	\item R --- результат работы программы, 7-разрядное беззнаковое число, т.к является счётчиком массива, длина которого ограничена $2^ 7-1$.  	
	\item $Y_{1..5}$ --- числа массива, 16-разрядные знаковые числа.
\end{itemize}

\subsection{Область допустимых значений}
\begin{itemize}
	\item Количество элементов массива --- может лежать в диапазоне $[0;127]$, т.к происходит прямая загрузка.
	\item Адрес начала массива --- это значение для конкретной программы - константа, но учитывая ограничение на длину массива в 127 элементов, расположение значений [359;35C] и программы [35D; 36D], можно сделать вывод, что адрес первого элемента массива может лежать в диапазоне $[0;358_{16}-127_{10}] \cup [\mathrm{36E}_{16};\mathrm{7FF}_{16}-127_{10}]$
	\item Результат --- т.к это счётчик, то максимальное значение - длина массива. Может принимать значения: $[0;127]$
\end{itemize}

\subsection{Расположение в памяти программы, исходных данных и результата}
\begin{itemize}
	\item Вспомогательные числа: 359 --- 35C 
	\item Программа: 35D --- 36D
	\item Массив: 36E --- 372
\end{itemize}

\subsection{Диапазон ячеек памяти, где может размещаться массив исходных данных}
\[
	[\mathrm{0};\mathrm{358}] \cup [\mathrm{36E}; \mathrm{7FF}]
\]